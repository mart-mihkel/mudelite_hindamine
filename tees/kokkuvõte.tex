\section{Kokkuvõte}
Töös leidsime vajalikud valimi suurused fikseeritud kindluse, veahinnangu lubatud maksimaalse suuruse ning mõnel juhul oletatud tegeliku kvaliteedimõõdu suhtes. Höffdingi võrratuse põhjal leitud tulemused ülehindasid vajaliku valimi suurust võrreldes binoomjaotuse omaduste põhjal arvutatud tulemustega, halvimal juhul isegi suurusjärgu võrra. Tehes arvutusi binoomjaotuse omaduste põhjal, pidime oletama kvaliteedimõõdu tegelikku väärtust. See-eest andis Höffdingi võrratus universaalse hinnangu, mis oli sõltumatu meetodi tegelikust kvaliteedimõõdust (binoomjaotuse parameetrist $p$).

Uurisime ka tehnikaid kahe masinõppe meetodi võrdlemiseks kvaliteedimõõtude põhjal. Uue mudeli kasutusele võtmisel oli küsimuseks kui palju see vanast parem on. Vastust küsimusele otsisime meetodite kvaliteedimõõtude vahede kaudu. Õigsuse ja täpsuse puhul oli võimalik vahele hinnangut leida märgendades manuaalselt vaid andepunkte, mille puhul meetodite klassifikatsioonid erinesid. Selle tulemusel oli võimalik vähendada märgendamist vajavate andmepunktide arvu. Saagise puhul oli osa lähendist avaldatav sarnaselt õigsusele ja täpsusele. Kuid saagise arvutamiseks on vajalik ikkagi teada kõikide andmepunktide tegelikke klassimärgendeid, seega on saagise hindamine märgendamise mõttes ikka raske.

Lisaks uurisime viise kuidas lähendada kvaliteedimõõtude vahesid komponentide kaupa. Definitsioonis kirjeldatud vahe eraldasime komponentideks, millest iga komponenti oli võimalik lähendada erinevate valimitega kasutades tulemusi eelnevatest peatükkidest. Tulemusena oli osa vahe hinnangust leitav täiesti märgendamata valimil. Saagise olemuse tõttu oli saagise hindamine märgendamise seisukohalt ikka kulukas.