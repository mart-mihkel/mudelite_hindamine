\section{Sissejuhatus}
Masinõppe mudelite hindamiseks kasutatakse erinevaid viise. Klassifitseerimisülesannete puhul saab ülesannet lahendava masinõppe meetodi headust hinnata kvaliteedimõõtudega. Kolm tihti kasutatud kvaliteedimõõtu on õigsus, täpsus ja saagis. Õigsus, täpsus ja saagis on nulli ja ühe vahelised numbrilised suurused, kus suurem väärtus tähendab paremat mudelit. Nimetatud suurused või nende hinnangud avalduvad mingi hulga andmepunktide tegelike klassimärgendite ja meetodi poolt klassifitseeritud klassimärgendite kaudu. Tegelike klassimärgendite leidmiseks peab need manuaalselt üle vaatama, mis võib paljude andmepunktide puhul muutuda kulukaks. Kvaliteedimõõtude täpsete väärtuste arvutamiseks peaks teadma populatsiooni kõigi andmepunktide märgendeid. Kuna kogu populatsiooni uurimine ei pruugi praktikas olla võimalik, leitakse sageli kvaliteedimõõtudele hinnangud üle lõpliku valimi.

Üle valimi arvutatud suuruste hinnangud ei ühti tavaliselt nende suuruste oodatud väärtusega, ehk üle populatsiooni arvutatud suurustega. Leitud hinnangud sisaldavad valimi juhuslikkuse tõttu vigu. Seega tuleb küsimuse alla kui suur antud viga on ning kui kindel saab olla, et viga ei ole suurem kui võib lubada. Antud töö käigus leitakse kui suurt valimit on vaja (või kui väikest valimit võib võtta), et mingi kindlusega ei ületaks hinnangu viga selle lubatud piiri.

Valimi puhul peab õigsuse, täpsuse või saagise hindamiseks definitsiooni põhjal leidma kõikide valimi andmepunktide märgendid. Ka minimaalse võimaliku valimi puhul võib see olla probleemne. Juhul kui asendada vana mudelit uuega ei pea küsima mis on uue meetodi õigsus, võib ka uurida kas ja kui palju on uue meetodi õigsus suurem vana omast. Selles töös on uuritud tehnikaid, mis aitavad vähendada märgendamist vajavate andmepunktida arvu kahe klassifitseerimismeetodi kvaliteedimõõtude vahede hindamiseks.

Esimese osas on toodud töös kasutatud mõisted ja definitsioonid ning kirjeldatud töö mõistmist abistavad taustteadmised. Töö teises osas on defineeritud klassifitseerimismeetodi kvaliteedimõõdud nende oodatud väärtuste kaudu, kirjeldatud viise nende hindamiseks ning uuritud kui suurt valimit on vaja küllaltki suure kindlusega piisavalt täpse hinnangu leidmiseks. Kolmandas osas uuritakse kuidas hinnata uut klassifitseerimismeetodit võrreldes seda juba olemasolevaga ning selle juures vältida kõikide andmepunktide märgendamist.