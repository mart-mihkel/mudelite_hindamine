\section{Sissejuhatus}
Masinõppe mudelite hindamiseks kasutatakse erinevaid viise. Klassifitseerimisülesannete puhul saab ülesannet lahendava masinõppe meetodi headust hinnata kvaliteedimõõtudega. Kolm tihti kasutatud kvaliteedimõõtu on õigsus, täpsus ja saagis. Õigsus, täpsus ja saagis on nulli ja ühe vahelised numbrilised suurused, kus suurem väärtus tähendab paremat mudelit. Need suurused või nende hinnangud avalduvad mingi hulga andmepunktide tegelike klassimärgendite ja meetodi poolt klassifitseeritud klassimärgendite kaudu. Tegelike klassimärgendite leidmiseks peame need manuaalselt üle vaatama, mis on suurte valimite puhul kulukas. Kvaliteedimõõtude täpsete väärtuste arvutamiseks peaksime teadma kogu populatsiooni andmepunktide märgendeid. Kuna kogu populatsiooni uurimine ei pruugi praktikas olla võimalik, leitakse sageli kvaliteedimõõtudele hinnangud üle lõpliku valimi.

Üle valimi arvutatud suuruste hinnangud ei ühti tavaliselt nende suuruste oodatud väärtusega, üle populatsiooni arvutatud suurustega. Leitud hinnangud sisaldavad valimi juhuslikkuse tõttu vigu. Seega võime küsida, kui suur on saadud viga ning kui kindlad saame olla, et viga ei ületa lubatud piiri. Selle töö käigus leiame kui suurt valimit vajame (või kui väike võib valim olla), et mingi kindlusega ei ületaks hinnangu viga selle lubatud piiri.

Valimi puhul peame õigsuse, täpsuse või saagise hindamiseks definitsiooni põhjal leidma kõikide valimi andmepunktide märgendid. Ka minimaalse võimaliku valimi puhul võib see olla probleemne. Juhul kui asendame vanat mudelit uuega, ei pea leidma uue meetodi õigsust, vaid võime uurida kas ja kui palju on uue meetodi õigsus suurem vana omast. Selles töös uurime tehnikaid, mis aitavad vähendada märgendamist vajavate andmepunktida arvu kahe klassifitseerimismeetodi kvaliteedimõõtude vahede hindamiseks.

Esimese osas toome välja töös kasutatud mõisted ja definitsioonid ning kirjeldame töö mõistmist aitavaid taustteadmisi. Töö teises osas defineerieemie klassifitseerimismeetodi kvaliteedimõõdud nende oodatud väärtuste kaudu, kirjeldame viise nende hindamiseks ning uurime kui suurt valimit vajame, et küllaltki suure kindlusega, saad piisavalt täpne hinnang. Kolmandas osas leiame kuidas hinnata uut klassifitseerimismeetodit võrreldes seda juba olemasolevaga ning selle juures vältida valimi kõikide andmepunktide märgendamist.