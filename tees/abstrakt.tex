% eesti keeles
\selectlanguage{estonian}

\noindent\textbf{\large Masinõppe mudelite hindamine väheste märgenditega andmetel}
\vspace*{1ex}
\noindent\textbf{Lühikokkuvõte:}
\noindent
Klassifitseerimisülesandeid lahendavate masinõppe mudelite hindamiseks kasutatakse kvaliteedimõõte nagu õigsus täpsus ja saagis. Need suurused või nende hinnangud avalduvad andmepunktide tegelike klassimärgendite ja meetodi klassifikatsioonide kaudu. Tegelike klassimärgendite leidmiseks peame need manuaalselt üle vaatama. Sageli hinnatakse kvaliteedimõõte üle lõpliku valimi, leitud hinnangud sisaldavad vigu. Antud töö käigus avaldame vajaliku valimi suuruse, et mingi kindlusega ei ületaks hinnangu viga selle lubatud piiri. Lisaks peame valimi puhul õigsuse, täpsuse või saagise hindamiseks definitsiooni põhjal leidma kõikide valimi andmepunktide märgendid. Kui lisaks hinnatavale klassifitseerimismeetodile on olemas teine meetod, saame seda kasutada esimese hindamiseks. Seejuures on võimalik märgendamiseks vajalikku manuaalset tööd vähendada, kui uurime uue meetodi kvaliteedimõõdu arvutamise asemel kui palju on uus meetod vanast parem. Töös uurime tehnikaid, mis aitavad vähendada märgendamist vajavate andmepunktida arvu kahe klassifitseerimismeetodi kvaliteedimõõtude vahede hindamiseks.
\vspace*{1ex}

\noindent\textbf{Võtmesõnad:} masinõpe, klassifitseerimine, tõenäosusteooria, statistika, õigsus, täpsus, saagis
\vspace*{1ex}

\noindent\textbf{CERCS:}P176 Tehisintellekt
\vspace*{3ex}

% inglise keeles
\selectlanguage{english}

\noindent\textbf{\large Evaluating machine learning models on data with few labels}
\vspace*{1ex}
\noindent\textbf{Abstract:}
\noindent
Machine learning models used to solve classification tasks are evaluated using quality measures such as accuracy, precision, and recall. These measures or their estimates are calculated using the class labels of data points and the classifications made by the method on those data points. To find the actual class labels, we must manually review the datapoints. Quality measures are often evaluated using a finite sample, and the obtained estimates contain errors. In this thesis we determine the necessary sample size for the estimates not to exceed a given limit of their error with a certain confidence level. The definition-based method for determining accuracy, precision, or recall over a given sample requires all the data points' labels to be known. If we have a second method alongside the one being evaluated, we can use it to asses the first. In this case, we can reduce the amount of manual work required for labeling by estiamting how much better the new method is compared to the old one. In this thesis we explore techniques which help reduce the number of data points that require labeling for evaluating the quality measures of the two classification methods.
\vspace*{1ex}

\noindent\textbf{Keywords:} machine learning, classification, probability theory, statistics, accruacy, precision, recall
\vspace*{1ex}

\noindent\textbf{CERCS:}P176 Artificial intelligence
\vspace*{1ex}

\selectlanguage{estonian}