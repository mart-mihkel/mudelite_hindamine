% eesti keeles
\selectlanguage{estonian}

\noindent\textbf{\large Masinõppe mudelite hindamine väheste märgenditega andmetel}
\vspace*{1ex}
\noindent\textbf{Lühikokkuvõte:} 
\noindent
Klassifitseerimisülesandeid lahendavate masinõppe mudelite hindamiseks kasutatakse kvaliteedimõõte nagu õigsus täpsus ja saagis. Nimetatud suurused või nende hinnangud avalduvad andmepunktide tegelike klassimärgendite ja meetodi klassifikatsioonide kaudu. Tegelike klassimärgendite leidmiseks peab need manuaalselt üle vaatama. Sageli hinnatakse kvaliteedimõõte üle lõpliku valimi, leitud hinnangud sisaldavad vigu. Antud töö käigus leiti kui suurt valimit on vaja, et mingi kindlusega ei ületaks hinnangu viga selle lubatud piiri. Lisaks peab valimi puhul õigsuse, täpsuse või saagise hindamiseks definitsiooni põhjal leidma kõikide valimi andmepunktide märgendid. Kui lisaks hinnatavale klassifitseerimismeetodile on olemas teine meetod, saab seda kasutada uue hindamiseks. Seejuures on võimalik märgendamiseks vajalikku manuaalset tööd vähendada, uurides uue meetodi kvaliteedimõõdu arvutamise asemel, kui palju on uus meetod vanast parem. Töös uuriti tehnikaid, mis aitavad vähendada märgendamist vajavate andmepunktida arvu kahe klassifitseerimismeetodi kvaliteedimõõtude vahede hindamiseks.
\vspace*{1ex}

\noindent\textbf{Võtmesõnad:} masinõpe, klassifitseerimine, tõenäosusteooria, statistika, õigsus, täpsus, saagis
\vspace*{1ex}

\noindent\textbf{CERCS:}P176 Tehisintellekt
\vspace*{3ex}

% inglise keeles
\selectlanguage{english}

\noindent\textbf{\large Evaluating machine learning models on data with few labels}
\vspace*{1ex}
\noindent\textbf{Abstract:}
\noindent
Machine learning models used to solve classification tasks are evaluated using quality measures such as accuracy, precision, and recall. These measures or their estimates are calculated through the class labels of data points and the classifications made by the method on those data points. To find the actual class labels, they must be manually reviewed. Often, quality measures are evaluated using a finite sample, and the obtained estimates may contain errors. In this thesis, the necessary sample size was determined not to exceed the limit of estimation error with a certain confidence level. Additionally, for a given sample, the definition-based method of determining accuracy, precision, or recall for all the sample data points' labels must be established. If another method exists alongside the one being evaluated, it can be used for a new assessment. In this case, it is possible to reduce the amount of manual work required for labeling by assessing how much better the new method is compared to the old one, rather than calculating the quality measures of the new method. This thesis explored techniques that help reduce the number of data points that require labeling for evaluating the quality measures of the two classification methods.
\vspace*{1ex}

\noindent\textbf{Keywords:} machine learning, classification, probability theory, statistics, accruacy, precision, recall
\vspace*{1ex}

\noindent\textbf{CERCS:}P176 Artificial intelligence
\vspace*{1ex}

\selectlanguage{estonian}